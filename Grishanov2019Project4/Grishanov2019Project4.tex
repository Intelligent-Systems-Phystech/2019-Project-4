\documentclass[12pt,twoside]{article}
\usepackage{jmlda}	
% https://docs.google.com/document/d/1zkOK6iLzuwVahOh138aTh5FzoX_LHCWlMiGS1LnsZhA/edit
\title
%    [Автоматическая настройка параметров BigARTM ] % Краткое название; не нужно, если полное название влезает в~колонтитул
    {Автоматическая настройка параметров BigARTM под широкий класс задач.}
\author
%    [Гришанов~А.\,В.] % список авторов для колонтитула; не нужен, если основной список влезает в колонтитул
    {Гришанов~А.\,В., Булатов~B.\,Г., Воронцов~К.\,В.} % основной список авторов, выводимый в оглавление
    [Гришанов~А.\,В.$^1$, Булатов~B.\,Г.$^1$, Воронцов~К.\,В.$^1$] % список авторов, выводимый в заголовок; не нужен, если он не отличается от основного
\thanks
%    {Работа выполнена при финансовой поддержке РФФИ, проект \No\,00-00-00000.
%   Научный руководитель:  Стрижов~В.\,В.
   {Задачу поставил:  Булатов~B.\,Г.
    Консультант:  Воронцов~К.\,В.}
\email
    {grishanov.av@phystech.ru, bt.uytya@gmail.com, vokov@forecsys.ru}
\organization
    {$^1$ Московский физико-технический институт}%; $^2$Организация}
\abstract
    {Открытая библиотека BigARTM позволяет строить тематические модели, используя широкий класс возможных регуляризаторов. При этом задача настройки коэффициентов оказывается весьма сложной. В данной статье мы ищем набор параметров, дающий <<достаточно хорошие>> результаты на широком классе задач, используя механизм относительных коэффициентов регуляризации и автоматический выбор N-грамм. Для эксперимента использовались наборы данных Victorian Era Authorship Attribution, 20 Newsgroups, МКБ-10. Модель с подобранными коэффициентами имеет качество не более чем на <X>\% хуже <<локально лучших моделей>>

\bigskip
\textbf{Ключевые слова}: \emph {тематическое моделирование, аддитивная регуляризация тематических моделей, PLSA, BigARTM}.}
\titleEng
    {JMLDA paper example: file jmlda-example.tex}
\authorEng
    {Author~F.\,S.$^1$, CoAuthor~F.\,S.$^2$, Name~F.\,S.$^2$}
\organizationEng
    {$^1$Organization; $^2$Organization}
\abstractEng
    {This document is an example of paper prepared with \LaTeXe\
    typesetting system and style file \texttt{jmlda.sty}.

    \bigskip
    \textbf{Keywords}: \emph{keyword, keyword, more keywords}.}
\begin{document}

\maketitle

\section{Введение}

Тематические модели широко используются на практике для решения задач классификации и ранжирования документов, а также для разведочного поиска\cite{Ianina2016}. Одним из распространённых инструментов в тематическом моделировании является библиотека bigARTM\cite{vorontsov2015bigartm}. Она предоставляет широкий выбор для настройки модели, используя обширный класс регуляризаторов. При этом возникает потребность в аккуратном подборе параметров.

ARTM имеет 5 типов параметров: $\alpha_1$, $\beta_1$ --- коэффициенты сглаживания для распределений тем в документах и для распределений терминов в темах, $\alpha_2$, $\beta_2$ --- коэффициенты разреживания распределений тем в документах и распределений терминов в темах, $\gamma$ --- коэффициент декоррелирования. 
Текущий подход к стратегии тюнинга параметров описан в работе \cite{Ianina2016}. Сначала подбирается один из регуляризатооров (например декоррелирования).  Его значение находится приблизительно, затем в зависимости от цели исследуется следующий параметр или продолжает улучшаться текущий. Существенно то, что имеются различные регуляризаторы  для различных итераций. Поэтому процесс перебора последовательный: добавляется один регуляризатор, оптимизируется, затем добавляется следующий. Из-за этого для оптимизации трудно использовать продвинутые методы, такие как байесовская оптимизация. Остаётся жадный или рандомизированный поиск.

Существующий процесс перебора не лишён недостатков. Используемые коэффициенты имеют характер абсолютных, а не относительных. Из-за этого у них появляются следующие негативные свойства. Во-первых абсолютные коэффициенты привязаны к одной коллекции и трудно переносятся на другие. Во-вторых, для них нет интуитивной интерпретации и советов по подбору. Стоит также отметить, что не существует единого критерия качества для тематической модели, это дополнительно усложняет задачу.

Цель данной работы --- исследовать процесс подбора параметров и найти критерии, по которым можно выбирать начальные параметры для широкого класса задач. Предлагается использовать относительные коэффициенты регуляризации и автоматический подбор N-граммм.

\section{Название раздела}

{\Huge TODO}

%\paragraph{Теоретическую часть работы} желательно структурировать
%с~помощью окружений
%Def, Axiom, Hypothesis, Problem, Lemma, Theorem, Corollary, State, Example, Remark.
%
%\begin{Def}
%    Математический текст \emph{хорошо структурирован},
%    если в~нём выделены определения, теоремы, утверждения, примеры, и~т.\,д.,
%    а~неформальные рассуждения (мотивации, интерпретации)
%    вынесены в~отдельные параграфы.
%\end{Def}
%
%\begin{State}
%    Мотивации и~интерпретации наиболее важны для понимания сути работы.
%\end{State}
%
%\begin{Theorem}
%    Не~менее $90\%$ коллег, заинтересовавшихся Вашей статьёй,
%    прочитают в~ней не~более~$10\%$ текста.
%\end{Theorem}
%
%\begin{Proof}
%    Причём это будут именно те~разделы, которые не содержат формул.
%\end{Proof}
%
%\begin{Remark}
%    Выше показано применение окружений
%    Def, Theorem, State, Remark, Proof.
%\end{Remark}
%

\section{Заключение}

{\Huge TODO}

%Желательно, чтобы этот раздел был, причём он не~должен дословно повторять аннотацию.
%Обычно здесь отмечают,
%каких результатов удалось добиться,
%какие проблемы остались открытыми.


\bibliographystyle{plain}
\bibliography{Grishanov2019Project4}
%\begin{thebibliography}{1}
%
%\bibitem{author09anyscience}
%    \BibAuthor{Author\;N.}
%    \BibTitle{Paper title}~//
%    \BibJournal{10-th Int'l. Conf. on Anyscience}, 2009.  Vol.\,11, No.\,1.  Pp.\,111--122.
%\bibitem{myHandbook}
%    \BibAuthor{Автор\;И.\,О.}
%    Название книги.
%    Город: Издательство, 2009. 314~с.
%\bibitem{author-and-co2007}
%    \BibAuthor{Автор\;И.\,О., Соавтор\;И.\,О.}
%    \BibTitle{Название статьи}~//
%    \BibJournal{Название журнала}. 2007. Т.\,38, \No\,5. С.\,54--62.
%\bibitem{bibUsefulUrl}
%    \BibUrl{www.site.ru}~---
%    Название сайта.  2007.
%\end{thebibliography}

% Решение Программного Комитета:
%\ACCEPTNOTE
%\AMENDNOTE
%\REJECTNOTE
\end{document}
